\documentclass[dvipdfmx,autodetect-engine,titlepage]{jsarticle}
\usepackage[dvipdfm]{graphicx}
\usepackage{ascmac}
\usepackage{fancybox}
\usepackage{listings}
\usepackage{plistings}
\usepackage{itembkbx}
\usepackage{amsmath}
\usepackage{svg}
\usepackage{url}
\usepackage{graphics}
\usepackage{listings,jvlisting}

\textheight=23cm
\renewcommand{\figurename}{図}
\renewcommand{\tablename}{表}
\newenvironment{code}
{\vspace{0.5zw}\VerbatimEnvironment  
\begin{screen} 
\baselineskip=1.0\normalbaselineskip
 \begin{Verbatim}}
{\end{Verbatim}
\baselineskip=\normalbaselineskip
 \end{screen}\vspace{0.5zw}} 

\title{情報理工学部 SNコース 3回\\
ワイヤレス通信システム\\
放射電磁界の距離依存性}
\author{2600200443-6\\Yamashita Kyohei\\山下 恭平}
\date{May 15 2022}

\begin{document}

\maketitle

\section{十分遠方における時の放射界の導出}

式(2・18)(2・20)から式(2・22)(2・23)を導出する。\\

\begin{align*}
  E_{\theta } &= \frac{Idl\sin \theta }{j4\pi \omega \epsilon}(\frac{k^2}{r}-\frac{jk}{r^2}-\frac{1}{r^3})e^{-jkr}\\
\end{align*}
十分遠方かつ真空の場合を考えているので。\\

\begin{align*}
  &= \frac{Idl\sin \theta }{j4\pi \omega \epsilon_0}\frac{k^2}{r}e^{-jkr}\\
\end{align*}

\begin{math}
  k^2 = \omega ^2 \epsilon \mu
\end{math}
より。

\begin{align*}
  &= \frac{Idl\omega \mu_0 \sin \theta }{j4\pi r}e^{-jkr}\\
\end{align*}

\begin{math}
  \omega = \frac{2\pi c}{\lambda }
\end{math}
より。

\begin{align*}
  &= \frac{Idl \mu_0 c\sin \theta }{j2\lambda r}e^{-jkr}\\
\end{align*}

\begin{math}
  c = \frac{1}{\sqrt{\epsilon _{0}\mu _0 } }
\end{math}
より

\begin{align*}
  &= \frac{Idl \mu_0 \sin \theta }{j2\lambda r \sqrt{\epsilon _{0}\mu _0 }}e^{-jkr}\\
\end{align*}

分母分子に
\begin{math}
  \sqrt{\mu _0} 
\end{math}
をかけると

\begin{align*}
  &= \frac{Idl\sin \theta }{j2\lambda r}\sqrt{\frac{\mu _0}{\epsilon _0} } e^{-jkr}\\
\end{align*}

\begin{math}
  \sqrt{\frac{\mu _0}{\epsilon _0} } = 120\pi
\end{math}
より

\begin{align*}
  &= \frac{j60\pi Idl\sin \theta }{\lambda r}e^{-jkr} \tag{(2・22)真ん中}
\end{align*}

同様に教科書、式(2・18)を計算すると。

\begin{align*}
  H_\phi &= \frac{Idl\sin \theta }{4\pi }(\frac{jk}{r}+\frac{1}{r^2})e^-jkr\\
  &= \frac{Idl\sin \theta }{4\pi }\frac{jk}{r}e^-jkr\\
  &= \frac{jIdl\omega \sqrt{\epsilon _{0}\mu _0}\sin \theta}{4\pi r}e^-jkr\\
  &= \frac{jIdl\sin \theta}{2\lambda r}e^-jkr
\end{align*}

となるので。

\begin{align*}
  E_{\theta } = \frac{j60\pi Idl\sin \theta }{\lambda r}e^{-jkr} = Z_{0}H_\phi \tag{2・22}\\
  H_\phi = \frac{jIdl\sin \theta}{2\lambda r}e^{-jkr} = \frac{E_\theta }{Z_0} \tag{2・23}
\end{align*}

\end{document}

