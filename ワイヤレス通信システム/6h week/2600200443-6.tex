\documentclass[dvipdfmx,autodetect-engine,titlepage]{jsarticle}
\usepackage[dvipdfm]{graphicx}
\usepackage{ascmac}
\usepackage{fancybox}
\usepackage{listings}
\usepackage{plistings}
\usepackage{itembkbx}
\usepackage{amsmath}
\usepackage{amssymb}
\usepackage{amsfonts}
\usepackage{svg}
\usepackage{url}
\usepackage{graphics}
\usepackage{listings,jvlisting}

\textheight=23cm
\renewcommand{\figurename}{図}
\renewcommand{\tablename}{表}
\newenvironment{code}
{\vspace{0.5zw}\VerbatimEnvironment  
\begin{screen} 
\baselineskip=1.0\normalbaselineskip
 \begin{Verbatim}}
{\end{Verbatim}
\baselineskip=\normalbaselineskip
 \end{screen}\vspace{0.5zw}} 

\title{情報理工学部 SNコース 3回\\
ワイヤレス通信システム\\
6th Week 演習問題}
\author{2600200443-6\\Yamashita Kyohei\\山下 恭平}
\date{Jun 4 2022}

\begin{document}

\maketitle

\section{問2}
開口の中心と開口の端部の距離さに相当する位相差が生じ、同位相では受信されない
から。

\section{問4}

\begin{align*}
  r_f=\frac{2D^2}{\lambda}
\end{align*}

であるので、f=800MHz,D=1.2,c=299792458の時

\begin{align*}
  r_f &= \frac{2.88}{\frac{299792458}{800}}\times10^{-6}\\
  &= 7.685316753365424\\
  &= 7.69
\end{align*}

よって,必要な距離は7.69m。\\\\

f=12GHzのとき

\begin{align*}
  r_f &= \frac{2.88}{\frac{299792458}{12}}\times10^{-9}\\
  &= 115.2797513004813\\
  &= 115.3
\end{align*}

よって,必要な距離は115.3m。


\section{問5}

指向利得性は以下の式で与えられる。

\begin{align*}
  G_{d}(\theta) = \frac{|D(\theta)^2 \vert }{\frac{1}{4\pi}\int_{0}^{2\pi}  \,d\phi\int_{0}^{\pi}|D(\theta)^2 \vert\sin\theta   \,d\theta  }
\end{align*}

これに、微小ダイポールアンテナの指向性係数\begin{math}
  D(\theta) = \sin\theta
\end{math}
を代入すると

\begin{align*}
  G_{d}(\theta) &= \frac{\sin^2\theta}{\frac{1}{4\pi}\int_{0}^{2\pi}  \,d\phi\int_{0}^{\pi}\sin^3\theta \, d\theta}\\
  &= \frac{4\pi\sin^2\theta}{2\pi\int_{0}^{\pi}\sin\theta(1-\cos^2  \theta)  \,d\theta }\\
  &= \frac{2\sin^2\theta}{\int_{0}^{\pi} \sin\theta(1-\frac{\cos2\theta+1}{2})\,d\theta }\\
  &= \frac{4\sin^4\theta}{\int_{0}^{\pi} \sin\theta - \sin\theta\cos2\theta\,d\theta}\\
  &= \frac{4\sin^4\theta}{\int_{0}^{\pi} \sin\theta - \frac{\sin3\theta-\sin\theta}{2} \,d\theta }\\
  &= \frac{8\sin^2\theta}{\int_{0}^{\pi} 3\sin\theta - \sin3\theta \,d\theta }\\
  &= \frac{8\sin^2\theta}{[-3\cos\theta+\frac{1}{3}\cos3\theta]_{0}^{\pi}}\\
  &= \frac{3}{2}\sin^2\theta
\end{align*}

よって,

\begin{align*}
  G_{d}(\theta) &= \frac{3}{2}\sin^2\theta
\end{align*}

\end{document}

