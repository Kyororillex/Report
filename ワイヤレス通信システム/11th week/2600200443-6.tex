\documentclass[dvipdfmx,autodetect-engine,titlepage]{jsarticle}
\usepackage[dvipdfm]{graphicx}
\usepackage{ascmac}
\usepackage{fancybox}
\usepackage{listings}
\usepackage{plistings}
\usepackage{itembkbx}
\usepackage{amsmath}
\usepackage{amssymb}
\usepackage{amsfonts}
\usepackage{svg}
\usepackage{url}
\usepackage{graphics}
\usepackage{listings,jvlisting}

\lstset{
  basicstyle={\ttfamily},
  identifierstyle={\small},
  commentstyle={\smallitshape},
  keywordstyle={\small\bfseries},
  ndkeywordstyle={\small},
  stringstyle={\small\ttfamily},
  frame={tb},
  breaklines=true,
  columns=[l]{fullflexible},
  numbers=left,
  xrightmargin=0zw,
  xleftmargin=3zw,
  numberstyle={\scriptsize},
  stepnumber=1,
  numbersep=1zw,
  lineskip=-0.5ex
}

\textheight=23cm
\renewcommand{\figurename}{図}
\renewcommand{\tablename}{表}
\newenvironment{code}
{\vspace{0.5zw}\VerbatimEnvironment  
\begin{screen} 
\baselineskip=1.0\normalbaselineskip
 \begin{Verbatim}}
{\end{Verbatim}
\baselineskip=\normalbaselineskip
 \end{screen}\vspace{0.5zw}} 

\title{情報理工学部 SNコース 3回\\
ワイヤレス通信システム\\
11th Week レポート}
\author{2600200443-6\\Yamashita Kyohei\\山下 恭平}
\date{Jul 10 2022}

\begin{document}

\maketitle

\section{4章 演習問題 問2}

\begin{align}
  V_1 = Z_{11}I_1 + Z_{12}I_2 + Z_{13}I_2 \\
  0 = Z_{21}I_1 + Z_{22}I_2 + Z_{23}I_2 \\
  0 = Z_{31}I_1 + Z_{32}I_2 + Z_{33}I_2 
\end{align}

(1)(2)(3)の両辺を\begin{math} I_1\end{math}で割ると

\begin{align}
  Z_{in} = \frac{V_1}{I_1} = Z_{11} + \frac{I_2}{I_1}Z_{12} + \frac{I_3}{I_1}Z_{13}\\
  Z_{21} + \frac{I_2}{I_1}Z_{22} + \frac{I_3}{I_1}Z_{23} = 0\\
  Z_{31} + \frac{I_2}{I_1}Z_{32} + \frac{I_3}{I_1}Z_{33} = 0
\end{align}

(4)(5)(6)において
\begin{align*}
  Z_{21} = Z_{12}\\
  Z_{31} = Z_{13}\\
  Z_{32} = Z_{23}
\end{align*}
であり、\begin{math}\frac{I_2}{I_1} = A , \frac{I_3}{I_1} = B\end{math}とすると

\begin{align}
  Z_{in} = Z_{11} + AZ_{12} + BZ_{13}\\
  Z_{12} + AZ_{22} + BZ_{23} = 0\\
  Z_{13} + AZ_{23} + BZ_{33} = 0
\end{align}

となる、よって式(8)(9)を解けば良い。\\
式(8)より

\begin{align}
  B = - \frac{Z_{12} + AZ_{22}}{Z_{23}}
\end{align}

(10)を(9)に代入すると

\begin{align}
  Z_{13} + AZ_{23} - \frac{Z_{12} + AZ_{22}}{Z_{23}}Z_{33} = 0\notag\\ 
  (Z_{23} - \frac{Z_{22}Z_{33}}{Z_{23}})A = \frac{Z_{12}Z_{33}}{Z_{23}} - Z_{13}\notag\\
  A = \frac{Z_{12}Z_{33}-Z_{13}Z_{23}}{Z_{23}^2-Z_{22}Z_{33}}
\end{align}

(10)と(11)より

\begin{align}
  B &= -\frac{Z_{12}}{Z_{23}} - \frac{Z_{22}}{Z_{23}}(\frac{Z_{12}Z_{33} - Z_{13}Z_{23}}{Z_{23}^2 - Z_{22}Z_{33}})\notag\\
  &= -\frac{1}{Z_{23}}(Z_{12} + \frac{Z_{12}Z_{22}Z_{33} - Z_{13}Z_{22}Z_{23}}{Z_{23}^2 - Z_{22}Z_{33}})\notag\\
  &= -\frac{Z_{12}Z_{23} - Z_{13}Z_{22}}{Z_{23}^2 - Z_{22}Z_{33}}
\end{align}

(7)(11)(12)より

\begin{align*}
  Z_{in} &= Z_{11} + \frac{Z_{12}Z_{33}-Z_{13}Z_{23}}{Z_{23}^2-Z_{22}Z_{33}}Z_{12} - \frac{Z_{12}Z_{23} - Z_{13}Z_{22}}{Z_{23}^2 - Z_{22}Z_{33}}Z_{13}\\
  &= Z_{11} + \frac{Z_{12}^2Z_{33} + Z_{13}^2Z_{22} - 2Z_{12}Z_{13}Z_{23}}{Z_{23}^2-Z_{22}Z_{33}}
\end{align*}

\section{4章 演習問題 問3}

軸比が1の時、円編波となるので

\begin{align*}
  |E_{\theta} \vert = |E_{\phi} \vert
\end{align*}

教科書式(4・13)より

\begin{align*}
  \lambda S &= 2\pi^{2}a^2\\
  S &= \frac{2\pi^{2}a^2}{\lambda}\\
  &= \frac{2 \times 3.1415 \times 4.5}{\frac{299792458}{1100}}\\
  &= 1.46
\end{align*}

よって、ピッチSが1.46mmの時、円偏波となる。

\end{document}

