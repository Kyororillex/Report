\documentclass[dvipdfmx,autodetect-engine]{jsarticle}
\usepackage[dvipdfm]{graphicx}
\usepackage{ascmac}
\usepackage{fancybox}
\usepackage{listings}
\usepackage{plistings}
\usepackage{itembkbx}
\usepackage{amsmath}
\usepackage{svg}
\usepackage{url}
\usepackage{graphics}
\usepackage{listings,jvlisting}
\usepackage{here}

\lstset{
  basicstyle={\ttfamily},
  identifierstyle={\small},
  commentstyle={\smallitshape},
  keywordstyle={\small\bfseries},
  ndkeywordstyle={\small},
  stringstyle={\small\ttfamily},
  frame={tb},
  breaklines=true,
  columns=[l]{fullflexible},
  numbers=left,
  xrightmargin=0zw,
  xleftmargin=3zw,
  numberstyle={\scriptsize},
  stepnumber=1,
  numbersep=1zw,
  lineskip=-0.5ex
}

\textheight=23cm
\renewcommand{\figurename}{図}
\renewcommand{\tablename}{表}
\newenvironment{code}
{\vspace{0.5zw}\VerbatimEnvironment  
\begin{screen} 
\baselineskip=1.0\normalbaselineskip
 \begin{Verbatim}}
{\end{Verbatim}
\baselineskip=\normalbaselineskip
 \end{screen}\vspace{0.5zw}} 

 \title{プロセスログの調査によるランサムウェアの動的解析} 
 \author{山下恭平、塚本覇虎、奥若菜}
 \date{2022年7月30日}
 \begin{document} 
 \maketitle

\section{実験の背景と目的}


\section{実行環境}

\begin{table}[h]
  \centering
  \begin{tabular}{|c|c|}
  \hline
  ソフトウェア名                         & バージョン          \\ \hline
  Oracle Virtual Box              & 6.1.32 r149290 \\ \hline
  Windows10 Enterprise Evaluation & 1809           \\ \hline
  Process Monitor                 & 3.89           \\ \hline
  \end{tabular}
  \end{table}

  \begin{figure}[H]
    \centering
    \fbox{\includegraphics[scale=0.6]{pic0.png}}
    \caption{実行環境図}
  \end{figure}

\end{document}

