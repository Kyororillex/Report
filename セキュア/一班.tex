\documentclass[dvipdfmx,autodetect-engine]{jsarticle}
\usepackage[dvipdfm]{graphicx}
\usepackage{ascmac}
\usepackage{fancybox}
\usepackage{listings}
\usepackage{plistings}
\usepackage{itembkbx}
\usepackage{amsmath}
\usepackage{svg}
\usepackage{url}
\usepackage{graphics}
\usepackage{listings,jvlisting}

\textheight=23cm
\renewcommand{\figurename}{図}
\renewcommand{\tablename}{表}
\newenvironment{code}
{\vspace{0.5zw}\VerbatimEnvironment  
\begin{screen} 
\baselineskip=1.0\normalbaselineskip
 \begin{Verbatim}}
{\end{Verbatim}
\baselineskip=\normalbaselineskip
 \end{screen}\vspace{0.5zw}} 

 \title{Process Monitorを用いたマルウェアの動的解析} 
 \author{山下恭平、塚本覇虎、奥若菜}
 \date{2022年5月17日}
 \begin{document} 
 \maketitle

\section{テーマ}
この実験では、擬似マルウェアツールShinoBotとProcess Monitorを用いた
マルウェアの動的解析を行う。

\section{環境}
本実験は以下の県境において実験を行う。
\begin{quote}
  \begin{itemize}
   \item Virtual Box ・・・ 仮想環境、今回はWindows10をインストールしている。
   \item Prosess Monitor ・・・ 動作しているプロセスを監視し、そのログを採取するソフト。
   \item ShinoBot ・・・        RAT(Random Access Tool)型のマルウェアを再現したソフト、web上に設置されたサーバから感染PCを操作できる。
  \end{itemize}
 \end{quote}

\section{行なったこと}
感染PCの特定のァイルをShinoBotから命令を送り削除した時のプロセスのログと、
通常の削除のログを取り、双方のログを比較しながら、怪しいプロセスの特定を
行った。

\section{分かったこと}
ShinoBotから送られてきた命令は「ShinoBot.exe → cmd.exe → cohost.exe → cmd.exe → powershell」
の順にプロセスが遷移し、削除が行われていた。\\
一方で通常の削除では、「cmd.exe → powershell」の順でプロセスが遷移しているのを確認した。
ShinoBotによるものが通常とは異なる挙動を示していたが、cmd.exeおよびpowershell
にて作成、実行されるプロセスはほとんどが一致していることも確認できた。

\section{予想}

\begin{quote}
  \begin{itemize}
   \item cohost.exeを一度経由することが、外部から操作されているときに見られる
   特徴である可能性。
  \end{itemize}
 \end{quote}

\end{document}

