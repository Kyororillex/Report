\documentclass[dvipdfmx,autodetect-engine,titlepage]{jsarticle}
\usepackage[dvipdfm]{graphicx}
\usepackage{ascmac}
\usepackage{fancybox}
\usepackage{listings}
\usepackage{plistings}
\usepackage{itembkbx}
\usepackage{amsmath}
\usepackage{svg}
\usepackage{url}
\usepackage{graphics}
\usepackage{listings,jvlisting}

\textheight=23cm
\renewcommand{\figurename}{図}
\renewcommand{\tablename}{表}
\newenvironment{code}
{\vspace{0.5zw}\VerbatimEnvironment  
\begin{screen} 
\baselineskip=1.0\normalbaselineskip
 \begin{Verbatim}}
{\end{Verbatim}
\baselineskip=\normalbaselineskip
 \end{screen}\vspace{0.5zw}} 

\title{情報理工学部 SNコース 3回\\
Webアプリケーション脆弱性演習レポート\\}
\author{2600200443-6\\Yamashita Kyohei\\山下 恭平}
\date{Apr 25 2021}

\begin{document}

\maketitle

\section{クロスサイト・スクリプティング}

\subsection{クロスサイト・スクリプティングとは}

\subsection{Level1演習}

\subsection{Level2演習}

\subsection{Level3演習}



\section{SQLインジェクション}

\subsection{SQLインジェクションとは}

\subsection{Level1演習}

\subsection{Level2演習}

\subsection{Level3演習}



\section{CSRF(クロスサイト・リクエスト・フォージェリ)}

\subsection{CSRFとは}

\subsection{Level1演習}

\subsection{Level2演習}



\section{OSコマンドイジェクション}

\subsection{OSコマンドイジェクションとは}

\subsection{Level1演習}

\subsection{Level2演習}


\end{document}

