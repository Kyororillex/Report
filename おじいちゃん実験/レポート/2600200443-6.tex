\documentclass[dvipdfmx,autodetect-engine,titlepage]{jsarticle}
\usepackage[dvipdfm]{graphicx}
\usepackage{ascmac}
\usepackage{fancybox}
\usepackage{listings}
\usepackage{plistings}
\usepackage{itembkbx}
\usepackage{amsmath}
\usepackage{amssymb}
\usepackage{amsfonts}
\usepackage{svg}
\usepackage{url}
\usepackage{graphics}
\usepackage{multirow}
\usepackage{listings,jvlisting}

\textheight=23cm
\renewcommand{\figurename}{図}
\renewcommand{\tablename}{表}
\newenvironment{code}
{\vspace{0.5zw}\VerbatimEnvironment  
\begin{screen} 
\baselineskip=1.0\normalbaselineskip
 \begin{Verbatim}}
{\end{Verbatim}
\baselineskip=\normalbaselineskip
 \end{screen}\vspace{0.5zw}} 

\title{情報理工学部 SNコース 3回\\
最終レポート\\}
\author{2600200443-6\\Yamashita Kyohei\\山下 恭平}
\date{Jul 17 2022}

\begin{document}

\maketitle

\section*{目的}

\section*{方法}

\section*{実験の内容}

\section*{問考察}

\section*{分析}

\section*{各回のレポート}

\subsection*{6月22日}

先週に続き、アンテナの改良をおこなった。先週はアンテナの長さを読売テレビに
合わせ約15cm後半から16cmとし、計測をおこなったが、今週はもう一度モデリング
を行い長さ15cmから約14.5cmまで縮めることにした。15cmの時は、全てのチャン
ネルが写りまた全てのチャンネルで抵抗を入れても写った。その中でも、毎日にテレ
ビは8dBの抵抗下でも正確に写すことができた。しかし、読売テレビは3dB、ABCは
4dBとアンテナの長さが読売テレビよりなのにも関わらず、読売テレビはあまり強
い強度で受信できていないことがわかった。次にアンテナの長さを14.5cmと少し
だけ短くしてもう一度実験を行った。結果は15cmの時に比べて、全体的に高い抵
抗値でも正確にテレビを映すことができた。中でもBBCは11dBの抵抗でも映すこ
とができ、読売テレビも4dBの抵抗まで耐えられるようになった。しかし、ABCテ
レビは2dBの抵抗までしか映すことができなかった。アンテナの長さを短くし、よ
り写りやすくなると予想していたが、異なる結果が得られてしまった。しかそ、こ
の時は時間があまりなく、急ぎでの測定だったこともあるので、自習にもう一度測定
をし直し、もう一度結果がどうなるのかを確認したいところである。15cm,14.5c
mの二つの測定結果から得られる考察を上げる。まず琵琶湖放送,BBCは地元のチャ
ンネルでもあることから、非常に強い電波強度で受信できていると考えられる。1
cmの方では8dB,14.5cmの方では11dBの抵抗下でも正常に映すことができた。対
照的に、ABC,読売テレビは、アンテナの性能影響を、比較的大きく受けているこ
とから、琵琶湖放送に比べ弱い電波強度であることが考えられる。だが、実験を
行っていて、持つアンテナの場所や角度で映るかどうかが決まっていた節も見ら
れたので、実験環境ももう一度整備し直し、実験を行うことでより正確なデータ
を採取したい。

\subsection*{6月29日}

先週に続き、アンテナの改良を行った。今回は、折り返しアンテナ、八木・宇多ア
ンテナ、ダイポールアンテナの後ろに金属板を設置したアンテナのモデリングを行
った。ダイポールアンテナの後ろに金属板を設置したアンテナについては、適切な
モデリングができ、計算の数値も比較的適切なものが得られたが、八木・宇多アン
テナのモデリングが、モデリングを行うツールの補正機能によって適切なモデリン
グ結果が得られなかった。しかし、演算結果としては非常に良い結果が得られたた
め、実際に発布スチロールと銅線を利用して、八木・宇多アンテナのさくせいを行
った。導波器と反射器の長さは適当に行い、その二つの距離を測定し、八歩スチロ
ールに固定した。テレビに接続し、測定を行った結果、すべてのチャンネルで4dB
以上の抵抗下でも映ることができたが、この結果は初めに作成した、ダイポール
アンテナとほとんど一致するものであった。これは、八木・宇多アンテナを作成
したのにも関わらず、あまり良い指向性が得られていなかったからだと考えられ
る。そのため、指向性を得るためのモデリングや、折り返しアンテナの作成を再
来週以降行っていきたい。

\subsection*{7月6日}

授業時間内で、先生へ実験の進捗報告をした。また、最終発表会に向けてチーム内
で左まっざまな準備を行った。私たちのグループは、VGA接続端子を持つコンピュ
ータを保持する学生がいないので、紙を映し出す装置を用いて最終発表を行うこ
とにした。先生への報告時に私たちの作成した八木宇多アンテナの間違っている
点について指摘された。その内容としては、私たちのグループが作成した八木宇
多アンテナは、モデリング上の八木宇多アンテナとは大きく縮尺が異なっていた
ことだ。モデリング上で作成した八木宇多アンテナの導波器と反射器の長さと私
たちが作成したものは大きく異なっていた。そのためか、実験結果としては、ダ
イポールアンテナと大きく変わらないものが出来上がってしまったとわかった。
そのため私たちはTAさんからMatlabの使い方について教わり、次週の実験までに
理想の八木宇多アンテナの導波器と反射器の長さを見つけるように、モデリング
行うことにした。Matalabでのモデリング結果をもとに次週での実験でアンテナ
を完成まで持っていきたい。

\subsection*{7月13日}



\end{document}