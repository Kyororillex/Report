\documentclass[dvipdfmx,autodetect-engine,titlepage]{jsarticle}
\usepackage[dvipdfm]{graphicx}
\usepackage{ascmac}
\usepackage{fancybox}
\usepackage{listings}
\usepackage{plistings}
\usepackage{itembkbx}
\usepackage{amsmath}
\usepackage{amssymb}
\usepackage{amsfonts}
\usepackage{svg}
\usepackage{url}
\usepackage{graphics}
\usepackage{multirow}
\usepackage{listings,jvlisting}

\textheight=23cm
\renewcommand{\figurename}{図}
\renewcommand{\tablename}{表}
\newenvironment{code}
{\vspace{0.5zw}\VerbatimEnvironment  
\begin{screen} 
\baselineskip=1.0\normalbaselineskip
 \begin{Verbatim}}
{\end{Verbatim}
\baselineskip=\normalbaselineskip
 \end{screen}\vspace{0.5zw}} 

\title{情報理工学部 SNコース 3回\\
リレーショナルデータベースにおけるモデリング\\}
\author{2600200443-6\\Yamashita Kyohei\\山下 恭平}
\date{Jun 19 2022}

\begin{document}

\maketitle

\section{データベースについて}

問題文中で与えられた表は、Exel的に表現おり、元のリレーショナルデータベースとは
異なる形をしているので、元の形に一度戻してみると、以下のようになると考えられる。
\begin{table}[h]
  \centering
  \begin{tabular}{cccccrrr}
  \hline
  \multicolumn{1}{|c|}{顧客ID}                  & \multicolumn{1}{c|}{顧客名}                   & \multicolumn{1}{c|}{所在地}                  & \multicolumn{1}{c|}{商品ID} & \multicolumn{1}{c|}{商品名}    & \multicolumn{1}{l|}{単価}  & \multicolumn{1}{l|}{個数}  & \multicolumn{1}{l|}{価格合計}                   \\ \hline\hline
  \multicolumn{1}{|c|}{A001}                  & \multicolumn{1}{c|}{大津電子}                  & \multicolumn{1}{c|}{大津市}                  & \multicolumn{1}{c|}{1010} & \multicolumn{1}{c|}{3mm ネジ} & \multicolumn{1}{r|}{10}  & \multicolumn{1}{r|}{300} & \multicolumn{1}{r|}{3000}                   \\ \hline
  \multicolumn{1}{|c|}{A001}                  & \multicolumn{1}{c|}{大津電子}                  & \multicolumn{1}{c|}{大津市}                  & \multicolumn{1}{c|}{1011} & \multicolumn{1}{c|}{丸形プラグ}  & \multicolumn{1}{r|}{200} & \multicolumn{1}{r|}{150} & \multicolumn{1}{r|}{30000}                  \\ \hline
  \multicolumn{1}{|c|}{A002}                  & \multicolumn{1}{c|}{草津精工}                  & \multicolumn{1}{c|}{草津市}                  & \multicolumn{1}{c|}{1010} & \multicolumn{1}{c|}{3mm ネジ} & \multicolumn{1}{r|}{10}  & \multicolumn{1}{r|}{200} & \multicolumn{1}{r|}{2000}                   \\ \hline
  \multicolumn{1}{|c|}{A002}                  & \multicolumn{1}{c|}{草津精工}                  & \multicolumn{1}{c|}{草津市}                  & \multicolumn{1}{c|}{1011} & \multicolumn{1}{c|}{丸形プラグ}  & \multicolumn{1}{r|}{200} & \multicolumn{1}{r|}{120} & \multicolumn{1}{r|}{24000}                  \\ \hline
  \multicolumn{1}{|c|}{A002}                  & \multicolumn{1}{c|}{草津精工}                  & \multicolumn{1}{c|}{草津市}                  & \multicolumn{1}{c|}{1045} & \multicolumn{1}{c|}{5m 銅線}  & \multicolumn{1}{r|}{500} & \multicolumn{1}{r|}{50}  & \multicolumn{1}{r|}{25000}                  \\ \hline
  \multicolumn{1}{l}{}                        & \multicolumn{1}{l}{}                       & \multicolumn{1}{l}{}                      & \multicolumn{1}{l}{}      & \multicolumn{1}{l}{}        & \multicolumn{1}{l}{}     & \multicolumn{1}{l}{}     & \multicolumn{1}{l}{}                        \\
                                              &                                            & \multicolumn{3}{c}{↓Exelなどによるセル結合}                                                                  & \multicolumn{1}{l}{}     & \multicolumn{1}{l}{}     & \multicolumn{1}{l}{}                        \\
                                              &                                            &                                           &                           &                             &                          &                          &                                             \\ \hline
  \multicolumn{1}{|c|}{顧客ID}                  & \multicolumn{1}{c|}{顧客名}                   & \multicolumn{1}{c|}{所在地}                  & \multicolumn{1}{c|}{商品ID} & \multicolumn{1}{c|}{商品名}    & \multicolumn{1}{l|}{単価}  & \multicolumn{1}{l|}{個数}  & \multicolumn{1}{l|}{価格合計}                   \\ \hline\hline
  \multicolumn{1}{|c|}{\multirow{2}{*}{A001}} & \multicolumn{1}{c|}{\multirow{2}{*}{大津電子}} & \multicolumn{1}{c|}{\multirow{2}{*}{大津市}} & \multicolumn{1}{c|}{1010} & \multicolumn{1}{c|}{3mm ネジ} & \multicolumn{1}{r|}{10}  & \multicolumn{1}{r|}{300} & \multicolumn{1}{r|}{\multirow{2}{*}{33000}} \\ \cline{4-7}
  \multicolumn{1}{|c|}{}                      & \multicolumn{1}{c|}{}                      & \multicolumn{1}{c|}{}                     & \multicolumn{1}{c|}{1011} & \multicolumn{1}{c|}{丸形プラグ}  & \multicolumn{1}{r|}{200} & \multicolumn{1}{r|}{150} & \multicolumn{1}{r|}{}                       \\ \hline
  \multicolumn{1}{|c|}{\multirow{3}{*}{A002}} & \multicolumn{1}{c|}{\multirow{3}{*}{草津精工}} & \multicolumn{1}{c|}{\multirow{3}{*}{草津市}} & \multicolumn{1}{c|}{1010} & \multicolumn{1}{c|}{3mm ネジ} & \multicolumn{1}{r|}{10}  & \multicolumn{1}{r|}{200} & \multicolumn{1}{r|}{\multirow{3}{*}{51000}} \\ \cline{4-7}
  \multicolumn{1}{|c|}{}                      & \multicolumn{1}{c|}{}                      & \multicolumn{1}{c|}{}                     & \multicolumn{1}{c|}{1011} & \multicolumn{1}{c|}{丸形プラグ}  & \multicolumn{1}{r|}{200} & \multicolumn{1}{r|}{120} & \multicolumn{1}{r|}{}                       \\ \cline{4-7}
  \multicolumn{1}{|c|}{}                      & \multicolumn{1}{c|}{}                      & \multicolumn{1}{c|}{}                     & \multicolumn{1}{c|}{1045} & \multicolumn{1}{c|}{5m 銅線}  & \multicolumn{1}{r|}{500} & \multicolumn{1}{r|}{50}  & \multicolumn{1}{r|}{}                       \\ \hline
  \end{tabular}
  \end{table}

この課題では、セル結合前の表をもとに考えることにする。

\section{どのような関数従属性が存在しているか}

\begin{align*}
  顧客ID \longrightarrow 顧客名 , 所在地 \\
  商品ID \longrightarrow 商品名 , 単価 \\
  顧客ID , 商品ID \longrightarrow 個数 , 合計価格
\end{align*}

\section{キーは何か}

\section{どの正規化のレベルを満たしているか}

\section{BCNFになるように分解せよ}

\section{BCNF に分解したあとのリレーションを利用して,「単価が 200 円以上の商品を販売した顧客の顧客名」を
得る問い合わせを高速に実行できるようにするには,どのリレーションのどの属性にどのような索引を構築
すればよいか。}

\section{「合計価格」の冗長性と、それに伴い発生する可能性の不整合についての検討}

\end{document}