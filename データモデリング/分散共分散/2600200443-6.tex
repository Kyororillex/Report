\documentclass[dvipdfmx,autodetect-engine,titlepage]{jsarticle}
\usepackage[dvipdfm]{graphicx}
\usepackage{ascmac}
\usepackage{fancybox}
\usepackage{listings}
\usepackage{plistings}
\usepackage{itembkbx}
\usepackage{amsmath}
\usepackage{amssymb}
\usepackage{amsfonts}
\usepackage{svg}
\usepackage{url}
\usepackage{graphics}
\usepackage{multirow}
\usepackage{listings,jvlisting}

\textheight=23cm
\renewcommand{\figurename}{図}
\renewcommand{\tablename}{表}
\newenvironment{code}
{\vspace{0.5zw}\VerbatimEnvironment  
\begin{screen} 
\baselineskip=1.0\normalbaselineskip
 \begin{Verbatim}}
{\end{Verbatim}
\baselineskip=\normalbaselineskip
 \end{screen}\vspace{0.5zw}} 

\title{情報理工学部 SNコース 3回\\
第三回レポート(ラグランジュの未定乗数法)\\}
\author{2600200443-6\\Yamashita Kyohei\\山下 恭平}
\date{Jul 14 2022}

\begin{document}

\maketitle

\section*{問題}

ラグランジュの未定乗数法を用い,周囲の長さが定数 L の長方形で.面積最大
のものは正方形であることを証明せよ.

\subsection*{問1 長方形の2辺をx, yとする.最大にするべき目的関数 f(x,y) を示せ.}

長方形の面積を求めれば良いので.

\begin{align}
  f(x,y) = xy
\end{align}

\subsection*{問2 長方形の2辺x, yが満たすべき制約式を示せ.}

長方形の全長がLであるので,満たすべき制約条件は.

\begin{align}
  2(x+y) = L
\end{align}

\subsection*{問3 Lagrange の未定乗数をλとして, x, y.λで偏微分するべき式をF(x, y, λ)で表わせ.}

(2)より,g(x,y)は

\begin{align}
  g(x,y) = 2(x+y) - L
\end{align}

となるので,求める式は(1),(3)より.

\begin{align}
  F(x,y,\lambda) &= f + \lambda g \notag\\
  &= xy + \lambda(2x+2y-L)
\end{align}

\subsection*{問4 式 F(x, y, λ)を x, y.λで偏微分し,その結果が0 とおいた方程式を書け.}

(4)について,x,y,\begin{math}\lambda\end{math}それぞれで偏微分を行うと.

\begin{align}
  \frac{\partial}{\partial x}F(x,y,\lambda) &= 0\notag\\
  y + 2\lambda &= 0
\end{align}

\begin{align}
  \frac{\partial}{\partial y}F(x,y,\lambda) &= 0\notag\\
  x + 2\lambda &= 0
\end{align}

\begin{align}
  \frac{\partial}{\partial \lambda}F(x,y,\lambda) &= 0\notag\\
  2x + 2y - L &= 0
\end{align}


\subsection*{問5 偏微分した結果が0 に等しいとき, x, yの関係を示せ.}

(5),(6),(7)よりx,yの関係は

\begin{align}
  x = y
\end{align}

ただし,

\begin{align*}
  x + y = \frac{L}{2}
\end{align*}

を満たす.\\
(8)より,周囲の長さが定数 L の長方形で.面積最大のものは正方形であることが
示された.

\end{document}