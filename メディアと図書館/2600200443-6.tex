\documentclass[dvipdfmx,autodetect-engine,titlepage]{jsarticle}
\usepackage[dvipdfm]{graphicx}
\usepackage{ascmac}
\usepackage{fancybox}
\usepackage{listings}
\usepackage{plistings}
\usepackage{itembkbx}
\usepackage{amsmath}
\usepackage{svg}
\usepackage{url}
\usepackage{graphics}
\usepackage{listings,jvlisting}

\textheight=23cm
\renewcommand{\figurename}{図}
\renewcommand{\tablename}{表}
\newenvironment{code}
{\vspace{0.5zw}\VerbatimEnvironment  
\begin{screen} 
\baselineskip=1.0\normalbaselineskip
 \begin{Verbatim}}
{\end{Verbatim}
\baselineskip=\normalbaselineskip
 \end{screen}\vspace{0.5zw}} 

\title{メディアと図書館\\
期末レポート\\
パンデミック終息後の経済活動について}

\author{情報理工学部 SNコース 2回\\
山下 恭平\\
学籍番号:26002004436}
\date{Jan 19 2022}

\begin{document}

\maketitle

\section{はじめに}
2020年に突然大流行した新型コロナウイルスによって私たちの生活は大きく変化した。
私はこのレポートを制作している時点で20年間世界を生きてきたわけだが、当然この
様なことは初めてである。今までの変化というのは、ゲーム機が徐々に小型化、高性能化
していったり、テレビが徐々に大きく、画質が良くなっていくなど時間を多く費やす
変化であったが、今回のコロナウイルスによるパンデミックでは、たった数ヶ月単位で
生活が変化していったのを鮮明に覚えている。そこで私は、自身に降りかかった変化だけではなく
、このパンデミックによって経済活動がどのように変化していくかに興味を持った。
このレポートではコロナ禍における経済活動を基に、パンデミック終息後の経済活動に
ついて考察を行っていく。

\section{コロナ禍における経済活動について}
2020年から現在にかけてまで、世界中で大流行している新型コロナウイルスの影響に
より、私たちの生活には様々な変化が生じていることは、私自身を含め非常に多くの
人が体感していることであろう。このパンデミックは経済活動に対しても大きな影響
を与えている、このことについて日経速報ニュース記事では以下のように記されている。

  \begin{quote}
    \begin{math}
      新型コロナウイルスの感染拡大で2020年の日本経済は大きく落ち込んだ。
      内閣府が昨年12月に公表した国民経済計算の20年度年次推計では、国内総生産(GDP)における
      暦年での付加価値の生産額を業種ごとにみることができる。コロナ禍は経済全体を
      押し下げたものの、「電気業」や「建設業」など業種によっては生産額を増やし、明暗が分かれた。^{(1)}
    \end{math}
  \end{quote}

この記事からも分かる様に、日本経済全体は落ち込んでいるが、一部の職種においては
逆に生産額が増えているという事実が見えてくる。私たちが、目に見えて経済活動
を停止した職種といえば、やはり飲食、観光業であろう、新型コロナウイルスの影響で
人間同士の接触を避けるようにする必要があり、飲食店では大きな入場制限、営業時間制限あるいは
一時的閉業、酒類の提供の禁止など、飲食店でのメリットを大きく損ねる対応を迫られた。
観光業においても、日本は外国人の入国を厳しく制限するだけでなく、日本国内の移動
までもを自粛するようにと強く訴えていたことは記憶に新しいだろう。その結果多くの観光地
は普段の賑わいを失い、多くの旅館、ホテル、娯楽施設などが厳しい状況へと追い込まれた。
この具体的状況について日本経済新聞の地方経済面よれば、
\begin{math}
  長野県のスノーリゾートでは、
感染拡大により修学旅行、社員研修、スキーの大会などが相次いで中止となったことで、少なく
とも3万6000人泊のキャンセルが発生し、損失額は3億円以上となり、観光客の入り込み
数は昨年と比べ7割以上減少した、^{(2)}
\end{math}
と記されている。しかし、飲食業、観光業などは人の流れが戻らない限り対策の仕様が
ないビジネスであるので、これらの職種についてはパンデミックの前後で特別大きな変化は
起きないと考えられる。また、建設業や電気業というのはウイルスの影響(Stay Home等)
によって、仕事が増えたものの、仕事の方法、形というのはパンデミックの前後で一切変化
しないと言えるだろう。では、パンデミックの前には無く、現在パンデミックの最中
に最も広がった経済活動はなんだろうかと聞かれれば、それは間違いなく「リモートワーク」
だろう。リモートワークを利用すれば会社員はオフィスに出社することなく、会議、開発、
といった業務を行うことが可能となる。社会人だけでなく、学生にも大きな影響を与えて
いることは現在大学生の私からすれば自明なことである。このパンデミックでリモートワー
クを導入した多くの企業が理解したことは、リモートワークでも会社は問題なく機能する
ということだろう。これは、パンデミック後の経済活動にとても大きく影響する。オフィス
にいかなくても良いということは場所が違うだけでなく、移動費や人件費、光熱費などの
経費を大きく削減することにもつながっている、しかし、リモートワークでの問題点も
もちろん存在する、その問題について武蔵野大学の渡部は以下の様に記している。

\begin{quote}
  \begin{math}
    しかしながら、マネジメントという観点に立てば、働く場所がオフィスではなく
    なることで生じる課題も浮かび上がってくる。同じ空間にいることによって部下
    の働きを目にすることができ、適時指示やアドバイスを与えていたマネジャーに
    とっては、文字通り異なる環境に置かれたことで、部下の様子を見ることができ
    ず、これまでと同じような言動では適切なマネジメントができなくなっている可能
    性がある。^{(3)}
  \end{math}
\end{quote}

オフィスからマネジメントする対象がいなくなるのだから、当然マネージャーはこの
問題にぶつかることになるだろう。しかし、オフィスで働いる人たちの、働き方の変化は
一方的なものでなく、相互に変化していくべきである。つまり、こういった問題を解決する
ことこそが、パンデミック後の経済活動に影響を与えると私は考えている。

\section{パンデミック収束後の経済活動についての考察}
一つ明らかなことは、パンデミック前と終息後では経済活動は大きく変化している
ということである。新型コロナウイルスによるパンデミックによって、多くの企業は
長年変更してこなかった働き方を強制的に考え直す必要があった、また、これまでは
何の問題もなく使用されていたソフトウェアやアプリケーションもそのユーザビリティ
を見つめ直す必要に迫られた。今回のパンデミックは多くの既存のものを見つめ直す
機会を与えたとも言えるだろう。では実際にどのように経済活動が変化していくかを
考えていく。まずは、飲食業、観光業などはパンデミック終息後、再び元の活気をとり
戻していくだろう。これらの職種については先ほども述べた通り、前と後で大きな変化は
見られないと考えられる、だが、飲食業については現在のコロナ禍において、デリバリーサー
ビスの需要が大きく上がったことによるデリバリー専門店の出現が最も大きな変化
だと考えられる。次にオフィスワーカーたちについて見ていく。パンデミック終息後に
オフィスワーカーの人たちはパンデミック以前の様に全員がオフィスに戻ってくるかと
聞かれれば、それは絶対にないと私は考えている。また、学校の講義についても、パンデミック
終息後にオンライン授業が廃止されるかと聞かれれば、それもないと考えられる。理由としては
、パンデミック中に新しく取り入れた、リモートワークなどの要素が想像以上に私たちの生活を
豊かにする可能性を秘めていたからである。好きな場所、環境を選択できるというのは、労働者、
学生にとってストレスを大きく軽減する一つの要因であることは、学生の私自身がよく感じている。
当然、一部の役職の人は常にオフィスにいる必要があり、出社する人と、リモートワークで参加
する人、といったようにハイブリット型が今後の主流になると私は予想している。ハイブリット
型にすることで、先ほど述べたマネージャーの問題もある程度解決するだろう。ここで私が
最も注目したいのは、なぜ、リモートワーク、リモート講義がこんなにも世の中に浸透した
のかである。先ほどは生活を豊かにする可能性があると述べたが、確かにその通りではあるが
、それを実現しているのは一体何なのかというと、IT業界の努力によってできていると言えるだろう。
新型コロナウイルスが拡大したことにより、オンライン会議ツールをはじめ、多くのソフトウェア、
サービスは自身のユーザビリティを見つめ直す必要に迫られた。ユーザビリティがいかに重要かに
ついて樽本は以下の様に述べている。

\begin{quote}
  \begin{math}
    ユーザビリティの訳語として「使いやすさ」がよく使われます。ただ、「ユーザビリティ = 
    使いやすさ」と捉えていると、"ユーザに対する思いやり"や"ユーザフレンドリ"といった
    主観的概念と混同してしまいます。  -(中略)-  しかし、ユーザビリティはもっと重大な
    意味を持っています。それは「使える」という意味です。つまり、「ユーザビリティに問題
    がある」ということは「使えない」という意味でもあるのです。^{(4)}
  \end{math}
\end{quote}

私たちが、ストレスなくリモートワークや、オンライン講義を利用できるのはユーザビリティが
しっかりと見つめ直されているからである。現在世界中で利用されているオンライン会議ツール
「Zoom」はパンデミック以前では誰も知らないサービスだったのに、パンデミック開始後
SkypeやTeamsを抑えてよく利用される様になったのか、これは間違いなくユーザビリティである。
Zoomはコロナ禍において求められているユーザビリティを最も正確に速く満たしたと考えられる。
パンデミック後に予想されるハイブリット型の経済活動でも、様々なソフトウェア、サービスが
利用されるだろう。これは、開発修正が行われるソフトウェア、サービスによって私たちの働き方が
大きく依存していると捉えることができる。これは、パンデミック前ではほとんど考えられなかった
ことだろう。いち早く、ハイブリット型に求められるユーザビリティを満たしたサービスは、非常に
様々な企業、学校で使用されることは簡単に想像することができる。この事実に、将来IT企業で働きたい
と考えている私は興味を持たざるを得ない。なぜなら、自身で生み出したサービスが一瞬にして世界で広まる
可能性をパンデミックによって知ることができたからである。これらのことからIT業界、情報系
学生はパンデミックの前後で持つ影響力が大きくなり、ユーザビリティがいかに有効かがより広まる
と考えられる。

\section{最後に}
パンデミック前後の経済活動について、「働き方」の観点から色々調べて、考察を行ったが
私が一番面白いと感じた部分は、今まで一切の変更を加えてこなかった経済活動が、パンデミックに
よって強制的に変更させられ、その結果うまくことが進みそう。というこの事実である、一見
確立された物事も、外部からの影響で破壊され再構築することでより良い方法が見つかるというのは、
経済活動だけでなく、日常的に生きていく上でとても重要なことだと感じた。また、今後SNSや、会議ツール
などの重要性、利便性が上がっていく中で、利便性の中に潜んだ危険性にも注目していきたいと感じた、例えば、
インターネットには大量の著作物が存在するが、それらの著作権を侵害するのはとても容易いことである。
著作権などを侵害しないためにも、技術の進歩とともに私たちユーザにもより高い教養を身につけることが求め
てられているということを、私たちは忘れてはならない。

\section{参考文献リスト}

(1) 「コロナとGDP 業種で明暗、電力など生産増 年次推計」, 『日経速報ニュース』, 2022/01/19\\
 \qquad\quad 日経テレコン21, \url{https://t21.nikkei.co.jp/g3/CMNDF11.do}, (2022/1/20)\\


(2) 「新型コロナ、修学旅行や社員研修中止、長野県内スキー場・温泉苦境、影響、「史上最悪」との声も。」\\
 \qquad\quad 『日本経済新聞』, 2020/3/11, 地方経済面, 長野, p3\\
 \qquad\quad 日経テレコン21, \url{https://t21.nikkei.co.jp/g3/CMNDF11.do}, (2022/1/20)\\

(3) 渡部 博志, 「リモートワークがもたらしたミドルマネジャーへの影響:コロナ禍
    におけるコミュニケーション機会の変化」, 『武蔵野大学経営研究所紀要』, 2021-09-01, 4号, p143 - p158\\

(4) 樽本 徹也, 『ユーザビリティエンジニアリング』, 「オーム 社」, 2014, 270p

\end{document}

